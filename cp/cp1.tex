\documentclass[12pt,letterpaper]{article}

%Note to Self:
%When Decommission? - after two months of feeling useless or right away?

\usepackage[utf8]{inputenc}
\usepackage[letterpaper,margin=1in]{geometry}
\usepackage{caption} % for table captions

\usepackage{amsmath} % for multi-line equations and piecewises
\usepackage{indentfirst} % to indent after a section
\usepackage{setspace}
\usepackage{times}
\usepackage{graphicx}
\usepackage{textcomp}
\usepackage{xspace}
\usepackage{verbatim} % for block comments
\usepackage{subfig} % for subfigures
\usepackage{enumitem} % for a) b) c) lists
\usepackage{tabularx}
\usepackage{cleveref}

\newcolumntype{b}{X}
\newcolumntype{s}{>{\hsize=.5\hsize}X}
\newcolumntype{m}{>{\hsize=.75\hsize}X}
\usepackage{titling}
\usepackage{minted}

\newcommand{\subtitle}[1]{%
  \posttitle{%
    \par\end{center}
    \begin{center}\large#1\end{center}
    \vskip0.5em}%
}
\usepackage{tikz}


\usetikzlibrary{shapes.geometric,arrows}
\tikzstyle{process} = [rectangle, rounded corners, minimum width=3cm, minimum height=1cm,text centered, draw=black, fill=blue!30]
\tikzstyle{arrow} = [thick,->,>=stealth]


\graphicspath{{images/}}
 
\usepackage[font={footnotesize,it}]{caption}
 



\setlength{\parindent}{15pt} % Default is 15pt.




\fontfamily{ptm}\selectfont

\title{CP1 for NPRE 501}
\author{Jin Whan Bae}
\date{2017-11-01}


\begin{document}
	
	\maketitle
	\hrule
	\onehalfspacing
	\thispagestyle{empty}

\section{Problem Definition}

\Cref{tab:constants} lists the constants used in the problem.


\begin{table}[h]
     \centering
    \begin{tabularx}{\textwidth}{bbb}
       \hline
       Parameter & Value & [Unit] \\
       \hline
       Diamater & 6 & [cm] \\
       \textbf{Radius} & \textbf{3} & [cm] \\
       Geometry & Sphere \\
       k & 15 & [$ \frac{W}{mK} $] \\
       Density & 8000 & [$ \frac{kg}{m^3} $] \\
       Specific Heat & 500 & [$ \frac{J}{kgK} $] \\
       \textbf{$ \alpha$} & \textbf{3.75e-6} & [$ \frac{m^2}{s} $] \\
       \hline
    \end{tabularx}
    \caption {Problem Constants. Derived constants are in bold.}
    \label{tab:constants}
\end{table}

Differential Equation:
\[\frac{1}{\alpha} \cdot \frac{dT}{dt} = \frac{1}{r^2} \frac{d}{dr} r^2 \frac{dT}{dr}\]

Boundary Conditions:
\[T(0,t) = finite \quad OR \quad \frac{dT}{dr} (r = 0, t) = 0\]
\[\frac{dT}{dr} (r = R) = 0 \]

Initial Condition:
\[T(r,0) = \frac{T_0}{2} (1-\cos{(\frac{\pi \cdot r}{R})}) \]

\section{C. Numerical Solution}

Expanding the differential equation:
\[\frac{1}{\alpha} \cdot \frac{dT}{dt} = \frac{2}{r} \frac{dT}{dr} + \frac{d^2T}{dr^2}\]

Applying the finite difference method, central for r and explicit for t:
\[\frac{1}{\alpha} \cdot \frac{T_k^{u+1} - T_k^{u}}{\Delta t} = \frac{2}{r_k} \; (\frac{T^u_{k+1}-T^u_{k-1}}{2 \Delta r}) + \frac{T^u_{k-1} - 2T^u_k + T^u_{k+1}}{\Delta r^2}\]

where u is the temporal step and k is the spacial step.

Solving for $T_k^u$:

\[T_k^u = \frac{\frac{2}{r_k} (\frac{T^u_{k+1}-T^u_{k-1}}{2 \Delta r}) + \frac{T^u_{k-1} + T^u_{k+1}}{\Delta r^2} - \frac{T_k^{u+1}}{\alpha \Delta t}}
{\frac{2}{\Delta r^2} - \frac{1}{\alpha \Delta t }}\]



\section{Appendix A}

\subsection{a. Final Temperature Distribution in the Sphere}
The final temperature distribution will be a cosine curve
with the highest point at r = 0, gradually going down to
the minimum value at r = R. 

\section{Appendix B}

\subsection{Analytical solution for solving T(r,t) directly}

\[\frac{1}{\alpha} \cdot \frac{dT}{dt} = \frac{1}{r^2} \frac{d}{dr} r^2 \frac{dT}{dr}\]

Boundary Conditions:
\[T(0,t) = finite \]
\[\frac{dT}{dr} (r = R) = 0 \]

Initial Condition:
\[T(r,0) = \frac{T_0}{2} (1-\cos{(\frac{\pi \cdot r}{R})}) \]

set:
\[T(r,t) = \frac{\overline{T}}{r} \]


\[\frac{1}{\alpha} \cdot \frac{d\overline{T}}{dt} \frac{1}{r} = \frac{1}{r^2} \frac{d}{dr} r^2 (\frac{d\overline{T}}{dr} \frac{1}{r} - \frac{1}{r^2} \overline{T}) \]

\[\frac{1}{\alpha} \cdot \frac{d\overline{T}}{dt} = \frac{1}{r} \frac{d}{dr} (\frac{d\overline{T}}{dr} r - \overline{T}) \]

\[\frac{1}{\alpha} \cdot \frac{d\overline{T}}{dt} = \frac{1}{r} (\frac{d^2\overline{T}}{dr^2} r + \frac{d\overline{T}}{dr} - \frac{d\overline{T}}{dr}) \]

\[\frac{1}{\alpha} \cdot \frac{d\overline{T}}{dt} = \frac{d^2\overline{T}}{dr^2} \]

turns into a cartesian problem.

Applying Separation of Variables:
\[\overline{T}(r,t) = \Gamma (t) \Psi (r)  \]

Boundary Conditions:

\[\Psi(r = 0) = finite \]
\[\frac{d\Psi}{dr} (r = R) = 0 \]

Applying the new variables, dividing both sides by $\Gamma (t) \Psi (r) $,
and setting it to a new variable $ -\beta^2 $:
\[\frac{1}{\alpha  \Gamma} \cdot \frac{d\Gamma}{dt} = \frac{d^2 \Psi}{dr^2} \frac{1}{\Psi} = -\beta^2 \] 

Solving for $\Psi$ first:
\[\frac{d^2 \Psi}{dr^2} \frac{1}{\Psi} = -\beta^2 \]

\[\frac{d^2 \Psi}{dr^2} + \beta^2 \Psi = 0 \]

\[\Psi (r) = C_1 sin(\beta r) + C_2 cos(\beta r) \]

Applying the first boundary condition,
interpreting as $\frac{d\Psi}{dr} (r = 0) = 0$, $C_1 = 0 $

Applying the second boundary condition,
\[0 =  -C_2 \beta sin(\beta R) \]
\[sin(\beta R) = 0  \]
\[\beta R = \pi, 2\pi, 3\pi ... \]
\[\beta_n = \frac{n\pi}{R} \quad where \quad n = 0 \; to \; \infty \]

Solving for $\Gamma(t)$:
\[\frac{1}{\alpha  \Gamma} \cdot \frac{d\Gamma}{dt} = -\beta_n^2 \]
\[\frac{1}{\alpha  \Gamma} \cdot \frac{d\Gamma}{dt} = -\beta_n^2 \alpha \Gamma \]

\[\Gamma(t) = A_1 e^{-\alpha \beta_n^2 t }\]

This makes $\overline{T}(r,t)$:
\[\overline{T}(r,t) = A_n e^{-\alpha \beta_n^2 t} sin(\beta_n r)\]
\[\beta_n = \frac{n\pi}{R} \quad where \quad n = 0 \; to \; \infty \]

Solving back for $T(r,t)$:
\[T(r,t) = \frac{A_n}{r} e^{-\alpha \beta_n^2 t} sin(\beta_n r)\]

Applying initial condition and orthogonality:
\[T(r,0) = \frac{A_n}{r} sin(\beta_n r) = \frac{T_0}{2} (1-\cos{(\frac{\pi \cdot r}{R})}) \]

\[A_n = \frac{\int_{0}^{R} \frac{T_0 r^3 \sin(\beta_n r)}{2} (1-\cos{(\frac{\pi \cdot r}{R})}) dr}{\int_{0}^{R} r^2 \sin^{2}(\beta_n r) dr}\]

The analytical solution is:
\[T(r,t) = \frac{\int_{0}^{R} \frac{T_0 r^3 \sin(\beta_n r)}{2} (1-\cos{(\frac{\pi \cdot r}{R})}) dr}{r \int_{0}^{R} r^2 \sin^{2}(\beta_n r) dr} e^{-\alpha \beta_n^2 t} sin(\beta_n r)\]
\[\beta_n = \frac{n\pi}{R} \quad where \quad n = 0 \; to \; \infty \]

\subsection{Code1}
\begin{minted}{xml}
<institution>
      <name>NO_ddd</name>
       <config><NO_ddd/></config>
      <reactor_list>
        <val>lwr</val>
        <val>sfr</val>
        <val>mox_lwr</val>
      </reactor_list>
\end{minted}


\end{document}






